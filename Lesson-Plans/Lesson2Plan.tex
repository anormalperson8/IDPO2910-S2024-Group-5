\documentclass{article}

\usepackage[a4paper, total={6in, 10.25in}]{geometry}
\usepackage{amsmath}
\usepackage{tikz}
\usepackage{silence}

\def\T{Teachers }
\def\t{teachers }
\def\S{Students }
\def\s{students }

\begin{document}
\begin{center}

\begin{tabular}{lllll}
Date: & 4$^{\text{th}}$ May, 2024 & \multicolumn{3}{l}{Time: 09:30 - 11:00 \& 14:15 - 15:45} \\
Venue: & Computer Barn C & \multicolumn{3}{l}{Expected no. of students: $\leq$ 25 per class} \\
\multicolumn{2}{l}{Expected Level of students: F.2 to F.4} & Context: & Coding/Programming in Python &\\
Foci: & \multicolumn{4}{l}{Boolean values, Loops, Lists, Game Creation}
\end{tabular}

\end{center}

\section*{Intended Learning Outcomes}

\begin{itemize}
\item[] By the end of the lesson, students should be able to:
\item manipulate lists to store and retrieve data.
\item work well with Python loops to repeat similar sets of instructions, or to iterate through lists.
\item create an medium-sized text-base game.
\end{itemize}

\section*{Basic Rundown}

\begin{itemize}
\item \T use Kahoot to do a short recap of previous lesson's concepts.
\item \T clear the misconceptions (if any).
\item \T demonstrate Hangman, and state the lesson's goal.
\item \T introduce the agenda of the lesson.
\item[] (The above takes $\leq$ 20 minutes)
\item \T teach the foci one-by-one.
	\begin{itemize}
	\item Boolean values
		\begin{itemize}
		\item \T expands more on Boolean values from previous lesson.
		\end{itemize}
	\item Lists ($\leq$ 15 minutes)
		\begin{itemize}
		\item \T introduce what lists are.
		\item \S practice mutating and accessing functions of lists with examples given.
		\item \S apply lists with loops and knowledge from previous lesson.
		\end{itemize}
	\item Loops ($\leq$ 15 minutes)
		\begin{itemize}
		\item \T introduce the \texttt{while} and \texttt{for} loops.
		\item \S practice loops with examples given.
		\end{itemize}
	\item \T summarise the above concepts.
	\item Final game (Remaining time)
		\begin{itemize}
		\item \T guide \s into implementing their own final game with the given Jupyter Notebook file. 
		\end{itemize}
	\end{itemize}
\end{itemize}

\section*{Materials}

\begin{itemize}
\item Computers at the venue to allow \s to have hands-on experience in programming.
\item A set of lecture notes to assist \t in the lesson and \s to follow along.
\item A set of Jupyter Notebook \texttt{.ipynb} files to allow \s to code along in the lesson.
\item Prizes (snacks, for example) for answering questions.
\end{itemize}

\end{document}