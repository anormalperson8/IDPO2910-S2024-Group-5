\documentclass{article}

\usepackage[a4paper, total={6in, 10.25in}]{geometry}
\usepackage{amsmath}
\usepackage{tikz}
\usepackage{silence}

\def\T{Teachers }
\def\t{teachers }
\def\S{Students }
\def\s{students }

\begin{document}
\begin{center}

\begin{tabular}{lllll}
Date: & 20$^{\text{th}}$ April & \multicolumn{3}{l}{Time: 09:30 - 11:00 \& 14:15 - 15:45} \\
Venue: & Computer Barn C & \multicolumn{3}{l}{Expected no. of students: 20 per class} \\
\multicolumn{2}{l}{Expected Level of students: F.2 to F.4} & Context: & Coding/Programming in Python &\\
Foci: & \multicolumn{4}{l}{I/O, Variables, Decision Making, \texttt{random} package, Game Creation}\\
\end{tabular}
\end{center}

\section*{Intended Learning Outcomes}

\begin{itemize}
	\item[] By the end of the lesson, \s should be able to/have:
	\item implement basic input/output statements in Python,
	\item manipulate data with variables in Python, combined with decision making,
	\item a very brief knowledge of the python \texttt{random} library and the usage of the \texttt{randint()} function,
	\item the skills to implement simple text-based operations in Python.
\end{itemize}


\section*{Basic Rundown}


\begin{itemize}
\item \T use Mentimeter to grasp students' knowledge background.

\item \T first demonstrate what the number guessing game is. Play it once with class.

\item \T introduce general knowledge around programming, Python and Jupyter Notebook

\item \T introduce the agenda of the lesson.

\item[] (The above takes $\leq$ 10 minutes)

\item \T teach the foci one-by-one.
	\begin{itemize}
		\item I/O and Variables ($\leq$ 25 minutes)
		\begin{itemize}
			\item \T introduce the \texttt{print()} function.
			\item \T introduces what a variable is, and basic data types and arithmetic operations.
			\item \S practice basic output and arithmetic operations.
			\item \T introduce the \texttt{input()} function.
			\item \T introduce type-casting methods.
			\item \S practice I/O with type-casting methods.
		\end{itemize}
		\item Python \texttt{random} package ($\leq$ 3 minutes)
		\begin{itemize}
			\item \T introduces the \texttt{random} package, particularly the \texttt{random.randint()} function.
		\end{itemize}
		\item Decision Making ($\leq$ 25 minutes)
		\begin{itemize}
			\item \T introduce \texttt{if}-\texttt{elif}-\texttt{else} clauses and conditions, including \texttt{or}, \texttt{and} (and possibly \texttt{not}) keywords.
			\item \S practice decision making with the number guessing game logic.
		\end{itemize}
		\item Finishing up the game (Remaining time)
		\begin{itemize}
			\item \T helps \s to finish up the game, and debug if necessary.
		\end{itemize}
	\end{itemize}

\item (If time \textit{really} permits, \t introduce a basic look at loops)

\item \T summarise the lesson, and tease what's to come, including loops, lists and the final game: Hangman.

\item[] (The above takes $\leq$ 5 minutes.)
\end{itemize} 

\section*{Materials}

\begin{itemize}
\item Access to computers at the venue to allow \s to have hands-on experience in programming.
\item A set of lecture notes to assist \t in the lesson and \s to follow along.
\item Small props (paper boxes, for example) to visualise certain concepts.
\item A Jupyter Notebook \texttt{.ipynb} file to allow \s to code along in the lesson.
\end{itemize}
\end{document}






















