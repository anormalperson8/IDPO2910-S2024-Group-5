\documentclass{article}

\usepackage[a4paper, total={6in, 10.25in}]{geometry}
\usepackage{amsmath}
\usepackage{tikz}
\usepackage{silence}

\def\T{Teachers }
\def\t{teachers }
\def\S{Students }
\def\s{students }

\begin{document}
\begin{center}

\begin{tabular}{lllll}
Date: & 20$^{\text{th}}$ April, 2024 & \multicolumn{3}{l}{Time: 09:30 - 11:00 \& 14:15 - 15:45} \\
Venue: & Computer Barn C & \multicolumn{3}{l}{Expected no. of students: 19 \& 24 per class} \\
\multicolumn{2}{l}{Expected Level of students: F.2 to F.4} & Context: & Coding/Programming in Python &\\
Foci: & \multicolumn{4}{l}{Programming, I/O, Variables, Decision Making}\\
\end{tabular}
\end{center}

\section*{Intended Learning Outcomes}

\begin{itemize}
	\item[] By the end of the lesson, \s should be able to/have:
	\item a basic understanding of what programming is,
	\item implement basic input/output statements in Python,
	\item manipulate data with variables in Python, combined with decision making,
	\item the skills to implement simple text-based operations in Python.
\end{itemize}


\section*{Basic Rundown}


\begin{itemize}
\item \T use Mentimeter to grasp students' knowledge background.

\item[] (The above takes $\leq$ 10 minutes)

\item \T teach the foci one-by-one.
	\begin{itemize}
		\item What is programming? ($\leq$ 15 minutes)
		\begin{itemize}
			\item \T introduces programming using real-life examples of tools and video games.
			\item \T introduce general knowledge around the programming language Python and Jupyter Notebook.
		\end{itemize}
		\item I/O and Variables ($\leq$ 20 minutes)
		\begin{itemize}
			\item \T introduces what a variable is, and basic data types and arithmetic operations.
			\item \S practice arithmetic operations.
			\item \T introduce the \texttt{print()} and \texttt{input()} functions.
		\end{itemize}
		\item Decision Making ($\leq$ 20 minutes)
		\begin{itemize}
			\item \T introduce \texttt{if}-\texttt{elif}-\texttt{else} clauses and conditions, including \texttt{or}, \texttt{and} and \texttt{not}) keywords.
			\item \S practice decision making with the number guessing game logic.
		\end{itemize}
	\end{itemize}

\item \T summarise the lesson, and tease what's to come, introduces a take-home exercise and the game: Hangman.

\item[] (The above takes $\leq$ 5 minutes.)
\end{itemize} 

\section*{Materials}

\begin{itemize}
\item Access to computers at the venue to allow \s to have hands-on experience in programming.
\item A set of lecture notes to assist \t in the lesson and \s to follow along.
\item A Jupyter Notebook (\texttt{.ipynb}) file to allow \s to code along in the lesson.
\end{itemize}
\end{document}






















