\documentclass[dvipsnames, svgnames, x11names, handout]{beamer}
\usepackage[T1]{fontenc}
\usepackage{bold-extra}
\usepackage{amsfonts, amsthm, amsmath, amssymb}
\usepackage{mathtools}
\usepackage{ulem}
\usepackage{bm}
\usepackage{minted}
\usemintedstyle{vs}
\usetheme{Madrid}
\usecolortheme{default}
\usepackage{multirow, multicol}
\usepackage{tikz, pgfplots}
\usetikzlibrary{arrows.meta, math, calc, quotes, intersections, angles, trees, shapes.geometric}
\pgfplotsset{compat=1.18}
\tikzstyle{block} = [rectangle, minimum width=4cm, minimum height=2cm, text centered, draw = black]
\tikzmath{\x1 = 2.5; \y1 = 3;}

\graphicspath{{../images}}

\title[HKUST Future-Ready Scholars]{HKUST Future-Ready Scholars}
\subtitle{Introduction to Game Programming using Python}
\author[Game Programming using Python]{Part 1: Number Guessing Game}
\date[April 2024]{20 April 2024}
\titlegraphic{\includegraphics[height=1cm]{ust.png}}

\begin{document}

\frame{\titlepage}

\begin{frame}[fragile]{Google Colab}
    \begin{center}
        We will use Google Colab for the workshops.

        \href{https://colab.research.google.com/}{https://colab.research.google.com/}

        You must have a Gmail account for it, create one if you do not.
    \end{center}
\end{frame}

\begin{frame}[fragile]{Files}
    \begin{center}
        All materials today are at:

        \href{https://bit.ly/ustidpo}{https://bit.ly/ustidpo}

        \

        \frame{\includegraphics[width=0.75\textwidth]{Download.png}}

        Download all files that belong to \textbf{Workshop 1} today.
    \end{center}
\end{frame}

\begin{frame}[fragile]{Jupyter Notebook}
    \begin{center}
        Now upload your Jupyter Notebook file with \textbf{Files $\rightarrow$ Open Notebook}.
    
        \

        \frame{\includegraphics[width=0.5\textwidth]{Upload.png}}

        Upload the file \textbf{Number-Guessing.ipynb}.
    
    \end{center}
\end{frame}

\begin{frame}[fragile]{Using Jupyter Notebook}
    You can type your code in these blocks. We call these blocks code cells.

    \begin{center}
        \frame{\includegraphics[width=0.5\textwidth]{Use1.png}}
    \end{center}

    \

    You can run a code cell with the button on the left.

    \begin{center}
        \frame{\includegraphics[width=0.5\textwidth]{Use2.png}}
    \end{center}
\end{frame}

\begin{frame}{World of Game Coding}
    \tikzstyle{block} = [rectangle, minimum width=4cm, minimum height=2cm, text centered, draw = black]
    \tikzmath{\x1 = 2.5; \y1 = 3;}
    
    \hspace{0.2\textwidth}\scalebox{0.5}{\begin{tikzpicture}
    \node (root) [block, fill = Cornsilk2] {Game (Software)};
    \node (program) [block, fill = LightSteelBlue] at ($(root) + (-\x1, -\y1)$) {Program (Codes)};
    \node (mm) [block, fill = Cornsilk2] at ($(root) + (\x1, -\y1)$) {Multimedia};
    
    \node (func) [block, fill = LightSteelBlue] at ($(program) + (0, -\y1)$) {Functions};
    
    \node (if) [block, fill = LightSteelBlue] at ($(func) + (-\x1, -\y1)$) {Decision Making};
    \node (loop) [block, fill = LightSteelBlue] at ($(func) + (\x1, -\y1)$) {Loops};
    
    \node (io) [block, fill = LightSteelBlue] at ($(if) + (0, -\y1)$) {Input/Output};
    \node (var) [block, fill = LightSteelBlue] at ($(loop) + (0, -\y1)$) {Variables};
    
    \draw [->] (root.south) -- ++(0, -0.5) -- ++(-\x1, 0) -- (program.north);
    \draw [->] (root.south) -- ++(0, -0.5) -- ++(\x1, 0) -- (mm.north);
    
    \draw [->] (program.south) -- (func.north);
    
    \draw [->] (func.south) -- ++(0, -0.5) -- ++(-\x1, 0) -- (if.north);
    \draw [->] (func.south) -- ++(0, -0.5) -- ++(\x1, 0) -- (loop.north);
    
    \draw [->] (loop.south) -- ++(0, -0.5) -- ++(-2*\x1, 0) -- (io.north);
    \draw [->] (if.south) -- ++(0, -0.5) -- ++(2*\x1, 0) -- (var.north);
    \end{tikzpicture}}
    
\end{frame}

\begin{frame}{What is Python?}
    Did you know? Python was made by someone who was bored.
    
    It's a language designed to be almost as understandable as English.
    
    You will be using Python 3. Why? Because Python 1 are 2 are too old.
    \begin{center}
    \includegraphics[height=1.5cm]{python-logo.png}\\
    This is the logo of Python.
    \end{center}
\end{frame}
    
\begin{frame}{Contents} 
\begin{center}
    \scalebox{0.5}{\begin{tikzpicture}
    \node (program) [block, fill = LightSteelBlue] {Program (Codes)};
    
    \node (func) [block, fill = LightSteelBlue] at ($(program) + (0, -\y1)$) {Functions};
    
    \node (if) [block, fill = LightSteelBlue] at ($(func) + (-\x1, -\y1)$) {Decision Making};
    \node (loop) [block, fill = LightSteelBlue] at ($(func) + (\x1, -\y1)$) {Loops};
    
    \node (io) [block, fill = DarkSeaGreen2] at ($(if) + (0, -\y1)$) {Input/Output};
    \node (var) [block, fill = LightSteelBlue] at ($(loop) + (0, -\y1)$) {Variables};
    
    \draw [->] (program.south) -- (func.north);
    
    \draw [->] (func.south) -- ++(0, -0.5) -- ++(-\x1, 0) -- (if.north);
    \draw [->] (func.south) -- ++(0, -0.5) -- ++(\x1, 0) -- (loop.north);
    
    \draw [->] (loop.south) -- ++(0, -0.5) -- ++(-2*\x1, 0) -- (io.north);
    \draw [->] (if.south) -- ++(0, -0.5) -- ++(2*\x1, 0) -- (var.north);
    \end{tikzpicture}}
\end{center}
\end{frame}

\begin{frame}[fragile]{The first thing in Python - \texttt{print()} function}

\begin{minted}[tabsize=4]{python} 
print("This is the print function.")
\end{minted}
\end{frame}
    
\begin{frame}[fragile]{The first thing in Python - \texttt{print()} function}
\texttt{print()} is a function that lets you print something,\\
also known as text output.
\begin{minted}[tabsize=4]{python}
print("Word") # This prints the word "Word".
\end{minted}
\vspace{1em}
Examples:
\begin{minted}[tabsize=4]{python} 
>>> print("Hello World")
Hello World
>>> print("Haha hehe")
Haha hehe
\end{minted}
\end{frame}

\begin{frame}[fragile]{Printing multiple things}
You can use a comma (\texttt{,}) to separate different things with a space.
\begin{minted}[tabsize=4]{python} 
>>> print("Alpha", "Beta", "Gamma")
Alpha Beta Gamma
>>> print("Haha", "hehe")
Haha hehe
\end{minted}
\end{frame}

\begin{frame}{Contents}

\begin{center}\scalebox{0.5}{
\begin{tikzpicture}
\node (program) [block, fill = LightSteelBlue] {Program (Codes)};

\node (func) [block, fill = LightSteelBlue] at ($(program) + (0, -\y1)$) {Functions};

\node (if) [block, fill = LightSteelBlue] at ($(func) + (-\x1, -\y1)$) {Decision Making};
\node (loop) [block, fill = LightSteelBlue] at ($(func) + (\x1, -\y1)$) {Loops};

\node (io) [block, fill = LightSteelBlue] at ($(if) + (0, -\y1)$) {Input/Output};
\node (var) [block, fill = DarkSeaGreen2] at ($(loop) + (0, -\y1)$) {Variables};

\draw [->] (program.south) -- (func.north);

\draw [->] (func.south) -- ++(0, -0.5) -- ++(-\x1, 0) -- (if.north);
\draw [->] (func.south) -- ++(0, -0.5) -- ++(\x1, 0) -- (loop.north);

\draw [->] (loop.south) -- ++(0, -0.5) -- ++(-2*\x1, 0) -- (io.north);
\draw [->] (if.south) -- ++(0, -0.5) -- ++(2*\x1, 0) -- (var.north);
\end{tikzpicture}}
\end{center}

\end{frame}

\begin{frame}[fragile]{Variables}
\begin{center}
Imagine you borrow a box from the computer.

\includegraphics[height=3cm]{box}

\pause
Give it a name and a value, you can now recall this value with the name!
\end{center}
\end{frame}

\begin{frame}[fragile]{Variables}
The code usually goes:\\
\texttt{variable\char`\_name = data}\\
This means whatever \texttt{data} is, it is now stored in a variable with name \texttt{variable\char`\_name}.
\vspace{1em}\\
Some basic variable types:
\begin{minted}[tabsize=4, escapeinside=||]{python}
a = 5       |\pause|# This is an integer (int) stored in a |\pause|
b = True    |\pause|# This is a boolean (bool) stored in b |\pause|
c = 3.2     |\pause|# This is a float (float)  stored in c |\pause|
d = "abc"   |\pause|# This is a string (str)   stored in d |\pause|
e = 'abc'   |\pause|# This is also a string    stored in e
\end{minted}
\end{frame}

\begin{frame}{ \ }
	\begin{center}
		The End\\
		Made in \LaTeX\\
		Last updated: 29 Mar 2024
	\end{center}
\end{frame}

\end{document}