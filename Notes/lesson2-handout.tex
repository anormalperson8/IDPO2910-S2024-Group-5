\documentclass[dvipsnames, svgnames, x11names, handout]{beamer}
\usepackage[T1]{fontenc}
\usepackage{bold-extra}
\usepackage{amsfonts, amsthm, amsmath, amssymb}
\usepackage{mathtools}
\usepackage{ulem}
\usepackage{minted}
\usemintedstyle{vs}
\usetheme{Madrid}
\usecolortheme{default}
\usepackage{multirow, multicol}
\usepackage{tikz, pgfplots}
\usetikzlibrary{arrows.meta, math, calc, quotes, intersections, angles, trees, shapes.geometric}
\pgfplotsset{compat=1.18}
\tikzstyle{block} = [rectangle, minimum width=4cm, minimum height=2cm, text centered, draw = black]
\tikzmath{\x1 = 2.5; \y1 = 3;}

\graphicspath{{../images}}

\title[Python Basics]{Coding in Python}
\subtitle{Python Basics - Part 2}
\date[May 2024]{4 May 2024}
\author[IDPO 2910 Group 5]{IDPO 2910 Group 5}
\titlegraphic{\includegraphics[height=1cm]{ust.png}}

\begin{document}
\frame{\titlepage}

\begin{frame}{Contents}
    \begin{center}\scalebox{0.5}{
        \begin{tikzpicture}
        \node (program) [block, fill = LightSteelBlue] {Program (Codes)};
        
        \node (func) [block, fill = LightSteelBlue] at ($(program) + (0, -\y1)$) {Functions};
        
        \node (if) [block, fill = LightSteelBlue] at ($(func) + (-\x1, -\y1)$) {Decision Making};
        \node (loop) [block, fill = LightSteelBlue] at ($(func) + (\x1, -\y1)$) {Loops};
        
        \node (io) [block, fill = LightSteelBlue] at ($(if) + (0, -\y1)$) {Input/Output};
        \node (var) [block, fill = DarkSeaGreen2] at ($(loop) + (0, -\y1)$) {Variables};
        
        \draw [->] (program.south) -- (func.north);
        
        \draw [->] (func.south) -- ++(0, -0.5) -- ++(-\x1, 0) -- (if.north);
        \draw [->] (func.south) -- ++(0, -0.5) -- ++(\x1, 0) -- (loop.north);
        
        \draw [->] (loop.south) -- ++(0, -0.5) -- ++(-2*\x1, 0) -- (io.north);
        \draw [->] (if.south) -- ++(0, -0.5) -- ++(2*\x1, 0) -- (var.north);
        \end{tikzpicture}}
    \end{center}
\end{frame}

\begin{frame}[fragile]{Lists}
Imagine you have a bunch of variables you want to store.
For example, if you have a bunch of people's names.
\begin{minted}[tabsize=4, escapeinside=||]{python}
name0 = "Chris Wong"
name1 = "Desmond Tsoi"
name2 = "Phoebe Mok"
name3 = "Nancy Ip"
\end{minted}
That is annoying to store and access.

\pause What if instead, we store it in the same thing, as a... \pause list?
\end{frame}

\begin{frame}[fragile]{Lists}
\begin{minted}[tabsize=4, escapeinside=||]{python}
names = ["Chris Wong", "Desmond Tsoi",
         "Phoebe Mok", "Nancy Ip"]
print(names[0]) # Chris Wong
\end{minted}

\

Lists are declared by surrounding the items with [ ], and separating each item with a comma.

We can access the name from a list by getting the corresponding item.

The first item in the list is the 0$^{\text{th}}$ item, second is 1$^{\text{st}}$ item, etc... 

We call this zero-indexing.\\

{\tiny Note: Some programming languages use one-indexing instead.\\[-1em]

\hspace{2.5em} If you approach another programming language, be careful.}
\end{frame}

\begin{frame}[fragile]{Lists}
\begin{minted}[tabsize=4, escapeinside=||]{python}
names = ["Chris Wong", "Desmond Tsoi",
         "Phoebe Mok", "Nancy Ip"]
print(names[0], names[1], names[2], names[3]) 
# Output:|\pause| Chris Wong Desmond Tsoi Phoebe Mok Nancy Ip
\end{minted}
\end{frame}

\begin{frame}[fragile]{Lists}
Another example:
\begin{minted}[tabsize=4, escapeinside=||]{python}
# Indices: 0  1  2  3  4  5
numbers = [0, 1, 1, 2, 3, 5]
print(numbers[0], numbers[1], numbers[2],
      numbers[3], numbers[4], numbers[5]) 
# Output: 0 1 1 2 3 5

print(numbers)
# Output: [0, 1, 1, 2, 3, 5]
\end{minted}

\begin{block}{Printing the whole list}
To print the whole list, simply put it in the \texttt{print()} function.
\end{block}
\end{frame}

\begin{frame}[fragile]{Lists}
To get the length of a list, we can use the \texttt{len()} function. \pause
\begin{minted}[tabsize=4, escapeinside=||]{python}
numbers = [0, 1, 1, 2, 3, 5]
print(len(numbers)) # 6
\end{minted}
\end{frame}

\begin{frame}[fragile]{Lists}
To edit an element of a list, assign the new value to the correct index. \pause
\begin{minted}[tabsize=4, escapeinside=||]{python}
numbers = [0, 1, 1, 2, 3, 5]
print(numbers) # [0, 1, 1, 2, 3, 5]
numbers[1] = 100 # Edit the second element (index 1)
print(numbers)  
# Output: |\pause| [0, 100, 1, 2, 3, 5]
\end{minted}
\end{frame}

\begin{frame}[fragile]{Lists}
To add an element to the end to a list, we use the \texttt{append(value)} list function. \pause
\begin{minted}[tabsize=4, escapeinside=||]{python}
numbers = [0, 1, 1, 2, 3, 5]
print(numbers, "length:", len(numbers)) 
# Output: [0, 1, 1, 2, 3, 5] length: 6
numbers.append(100) # Add 100 to the end of the list
print(numbers, "length:", len(numbers))
# Output:|\pause| [0, 1, 1, 2, 3, 5, 100] length: 7
\end{minted}
\end{frame}

\begin{frame}[fragile]{Lists}
To insert an element to a particular position in a list, we use the \texttt{insert()} list function.

The \texttt{insert(i, value)} inserts the \texttt{value} at index \texttt{i}, and push everything after to the right. \pause
\begin{minted}[tabsize=4, escapeinside=||]{python}
numbers = [0, 1, 1, 2, 3, 5]
print(numbers, "length:", len(numbers)) 
# Output: [0, 1, 1, 2, 3, 5] length: 6
numbers.insert(2, 100) # Add 100 to index 2 of the list
print(numbers, "length:", len(numbers))  
# Output:|\pause| [0, 1, 100, 1, 2, 3, 5] length: 7
numbers.insert(7, 200) # Same as numbers.append(200)
print(numbers, "length:", len(numbers))  
# Output:|\pause| [0, 1, 100, 1, 2, 3, 5, 200] length: 8
\end{minted}
\end{frame}

\begin{frame}[fragile]{Lists}
To remove an element from a list, we use the \texttt{remove()} list function.

The \texttt{remove(value)} function removes the \textbf{first} occurence of \texttt{value}. \pause
\begin{minted}[tabsize=4, escapeinside=||]{python}
numbers = [0, 1, 1, 2, 3, 5]
print(numbers, "length:", len(numbers)) 
# Output: [0, 1, 1, 2, 3, 5] length: 6
numbers.remove(1) # Remove the first occurence of number 1
print(numbers, "length:", len(numbers))
# Output:|\pause| [0, 1, 2, 3, 5] length: 5
\end{minted}
\end{frame}

\begin{frame}[fragile]{Lists}
The \texttt{reverse()} list function reverses a list's contents. \pause

\begin{minted}[tabsize=4, escapeinside=||]{python}
numbers = [0, 1, 1, 2, 3, 5]
print(numbers, "length:", len(numbers)) 
# Output: [0, 1, 1, 2, 3, 5] length: 6
numbers.reverse() # Reverse the list
print(numbers, "length:", len(numbers)) 
# Output:|\pause| [5, 3, 2, 1, 1, 0] length: 6
print(numbers[0])
# Output:|\pause| 5
\end{minted}
\end{frame}

\begin{frame}[fragile]{Lists}
The \texttt{count(item)} list function counts the number of occurence of \texttt{item} in a list. \pause

\begin{minted}[tabsize=4, escapeinside=||]{python}
numbers = [0, 1, 1, 2, 3, 5]
print(numbers.count(1)) 
# Output:|\pause| 2
print(numbers.count(100)) 
# Output:|\pause| 0
\end{minted}
\end{frame}

\begin{frame}[fragile]{Lists}
The \texttt{index(item)} list function finds the index of the first occurence of \texttt{item} in a list. \pause

\begin{minted}[tabsize=4, escapeinside=||]{python}
numbers = [0, 1, 1, 2, 3, 5]
print(numbers.index(1)) 
# Output:|\pause| 1
print(numbers.index(5)) 
# Output:|\pause| 5
print(numbers.index(100)) |\pause|
# Output: No output, error, 100 is not in the list
\end{minted}
\end{frame}

\begin{frame}[fragile]{Lists}
The \texttt{sort()} list function sorts a list's contents. \pause

\begin{minted}[tabsize=4, escapeinside=||]{python}
numbers = [6, 5, 1, 2, 3]
print(numbers, "length:", len(numbers)) 
# Output: [6, 5, 1, 2, 3] length: 5
print(numbers[0])
# Output: 6
numbers.sort() # Sort the list
print(numbers, "length:", len(numbers)) 
# Output:|\pause| [1, 2, 3, 5, 6] length: 5
print(numbers[0])
# Output:|\pause| 1
\end{minted}
\end{frame}

\begin{frame}{Contents}
    \begin{center}\scalebox{0.5}{
        \begin{tikzpicture}
        \node (program) [block, fill = LightSteelBlue] {Program (Codes)};
        
        \node (func) [block, fill = LightSteelBlue] at ($(program) + (0, -\y1)$) {Functions};
        
        \node (if) [block, fill = LightSteelBlue] at ($(func) + (-\x1, -\y1)$) {Decision Making};
        \node (loop) [block, fill = DarkSeaGreen2] at ($(func) + (\x1, -\y1)$) {Loops};
        
        \node (io) [block, fill = LightSteelBlue] at ($(if) + (0, -\y1)$) {Input/Output};
        \node (var) [block, fill = LightSteelBlue] at ($(loop) + (0, -\y1)$) {Variables};
        
        \draw [->] (program.south) -- (func.north);
        
        \draw [->] (func.south) -- ++(0, -0.5) -- ++(-\x1, 0) -- (if.north);
        \draw [->] (func.south) -- ++(0, -0.5) -- ++(\x1, 0) -- (loop.north);
        
        \draw [->] (loop.south) -- ++(0, -0.5) -- ++(-2*\x1, 0) -- (io.north);
        \draw [->] (if.south) -- ++(0, -0.5) -- ++(2*\x1, 0) -- (var.north);
        \end{tikzpicture}}
    \end{center}
\end{frame}

\begin{frame}[fragile]{Loops}
What do you do if you want to do something repeatedly in code? \pause 
\begin{minted}[tabsize=4, escapeinside=||]{python}
print("Count:", 10)
print("Count:", 9)
print("Count:", 8)
print("Count:", 7)
print("Count:", 6)
print("Count:", 5)
print("Count:", 4)
print("Count:", 3)
print("Count:", 2)
print("Count:", 1)
print("Done.")
\end{minted} 
\pause 
Let's turn this into a loop.
\end{frame}

\begin{frame}[fragile]{Loops - \texttt{while}}
Example:
\begin{minted}[tabsize=4, escapeinside=||]{python}
i = 10 # Initialising i as 10
while i > 0:
    print("Count:", i)
    i = i - 1
print("Done.")
\end{minted}
Let's run through it together.
\end{frame}

\addtocounter{framenumber}{-1}

\begin{frame}[fragile]{Loops - \texttt{while}}
Example:
\begin{minted}[tabsize=4, escapeinside=||]{python}
i = 10
while i > 0: # i is 10, which is larger than 0
    print("Count:", i)
    i = i - 1
print("Done.")
\end{minted}
\end{frame}

\addtocounter{framenumber}{-1}

\begin{frame}[fragile]{Loops - \texttt{while}}
Example:
\begin{minted}[tabsize=4, escapeinside=||]{python}
i = 10
while i > 0: 
    print("Count:", i) # Count: 10
    i = i - 1
print("Done.")
\end{minted}
\end{frame}

\addtocounter{framenumber}{-1}

\begin{frame}[fragile]{Loops - \texttt{while}}
Example:
\begin{minted}[tabsize=4, escapeinside=||]{python}
i = 10
while i > 0: 
    print("Count:", i)
    i = i - 1 # i goes from 10 to 9, then we go back up
print("Done.")
\end{minted}
\end{frame}

\addtocounter{framenumber}{-1}

\begin{frame}[fragile]{Loops - \texttt{while}}
Example:
\begin{minted}[tabsize=4, escapeinside=||]{python}
i = 10
while i > 0: # i is 9, which is larger than 0
    print("Count:", i)
    i = i - 1
print("Done.")
\end{minted}
\end{frame}

\addtocounter{framenumber}{-1}

\begin{frame}[fragile]{Loops - \texttt{while}}
Example:
\begin{minted}[tabsize=4, escapeinside=||]{python}
i = 10
while i > 0: 
    print("Count:", i) # Count: 9
    i = i - 1
print("Done.")
\end{minted}
\end{frame}

\addtocounter{framenumber}{-1}

\begin{frame}[fragile]{Loops - \texttt{while}}
Example:
\begin{minted}[tabsize=4, escapeinside=||]{python}
i = 10
while i > 0: 
    print("Count:", i)
    i = i - 1 # i goes from 9 to 8, then we go back up
print("Done.")
\end{minted}
\end{frame}

\addtocounter{framenumber}{-1}

\begin{frame}[fragile]{Loops - \texttt{while}}
Example:
\begin{minted}[tabsize=4, escapeinside=||]{python}
i = 10
while i > 0: # i is 8, which is larger than 0
    print("Count:", i)
    i = i - 1
print("Done.")
\end{minted}
\end{frame}

\addtocounter{framenumber}{-1}

\begin{frame}[fragile]{Loops - \texttt{while}}
Example:
\begin{minted}[tabsize=4, escapeinside=||]{python}
i = 10
while i > 0: 
    print("Count:", i) # Count: 8
    i = i - 1
print("Done.")
\end{minted}
\end{frame}

\addtocounter{framenumber}{-1}

\begin{frame}[fragile]{Loops - \texttt{while}}
Example:
\begin{minted}[tabsize=4, escapeinside=||]{python}
i = 10
while i > 0: 
    print("Count:", i)
    i = i - 1 # i goes from 8 to 7, then we go back up
print("Done.")
\end{minted}
\end{frame}

\addtocounter{framenumber}{-1}

\begin{frame}[fragile]{Loops - \texttt{while}}
Example:
\begin{minted}[tabsize=4, escapeinside=||]{python}
i = 10
while i > 0: # i is 7, which is larger than 0
    print("Count:", i)
    i = i - 1
print("Done.")
\end{minted}

This goes on and on\dots
\end{frame}

\addtocounter{framenumber}{-1}

\begin{frame}[fragile]{Loops - \texttt{while}}
Example:
\begin{minted}[tabsize=4, escapeinside=||]{python}
i = 10
while i > 0: # i is 1, which is larger than 0
    print("Count:", i)
    i = i - 1
print("Done.")
\end{minted}
\end{frame}

\addtocounter{framenumber}{-1}

\begin{frame}[fragile]{Loops - \texttt{while}}
Example:
\begin{minted}[tabsize=4, escapeinside=||]{python}
i = 10
while i > 0: 
    print("Count:", i) # Count: 1
    i = i - 1
print("Done.")
\end{minted}
\end{frame}

\addtocounter{framenumber}{-1}

\begin{frame}[fragile]{Loops - \texttt{while}}
Example:
\begin{minted}[tabsize=4, escapeinside=||]{python}
i = 10
while i > 0: 
    print("Count:", i)
    i = i - 1 # i goes from 1 to 0, then we go back up
print("Done.")
\end{minted}
\end{frame}

\addtocounter{framenumber}{-1}

\begin{frame}[fragile]{Loops - \texttt{while}}
Example:
\begin{minted}[tabsize=4, escapeinside=||]{python}
i = 10
while i > 0: # i is 0, which is NOT larger than 0, so we exit
    print("Count:", i)
    i = i - 1
print("Done.")
\end{minted}
\end{frame}

\addtocounter{framenumber}{-1}

\begin{frame}[fragile]{Loops - \texttt{while}}
Example:
\begin{minted}[tabsize=4, escapeinside=||]{python}
i = 10
while i > 0:
    print("Count:", i)
    i = i - 1
print("Done.") # "Done." is printed
\end{minted}
\end{frame}

\begin{frame}[fragile]{Loops - \texttt{for}}
Example:
\begin{minted}[tabsize=4, escapeinside=||]{python}
for i in range(10):
    print("Count:", i)
print("Done.")
\end{minted}
\begin{block}{Python \texttt{range}}
Python \texttt{range} is a thing of mystery. When you do \texttt{range(n)}, where \texttt{n} is an integer, Python generates a \textit{range} of integers from 0 to \texttt{n} - 1.
\end{block}
\end{frame}

\begin{frame}[fragile]{Loops - \texttt{for}}
Not getting the loop?
\begin{minted}[tabsize=4, escapeinside=||]{python}
for i in range(10):
    print("Count:", i)
print("Done.")
\end{minted}

\

is equivalent to

\

\begin{minted}[tabsize=4, escapeinside=||]{python}
i = 0
while i < 10:
    print("Count:", i)
    i = i + 1
print("Done.")
\end{minted}

Both loops go from 0 to 10, and give identical output.
\end{frame}

\begin{frame}[fragile]{Loops - \texttt{for}}
Another example:
\begin{minted}[tabsize=4, escapeinside=||]{python}
for i in range(5):
    print(i*i, end=" ") # Print the square, end with a space
print() # Add new line at the end
# Output:|\pause| 0 1 4 9 16
\end{minted}
\end{frame}

\begin{frame}[fragile]{Loops - \texttt{for}}
Let's combine lists with a for loop.
\begin{minted}[tabsize=4, escapeinside=||]{python}
num = [1, 5, 9, 12, 4, 800]
for i in range(len(num)):
    print(num[i], end=" ") # Print num[i], end with a space
print() # Add new line at the end
# Output:|\pause| 1 5 9 12 4 800
\end{minted}
\pause This is how we go through a list.
\end{frame}

\begin{frame}[fragile]{Loops - \texttt{for}}
Instead of using the index, there is another way to go through a list:
\begin{minted}[tabsize=4, escapeinside=||]{python}
num = [1, 5, 9, 12, 4, 800]
for i in num:
    print(i, end=" ")
print() # Add new line at the end
# Output:|\pause| 1 5 9 12 4 800
\end{minted}
\pause The output is identical to the previous example.
\end{frame}

\begin{frame}[fragile]{Loops - \texttt{for}}
It also works for lists of other types.
\begin{minted}[tabsize=4, escapeinside=||]{python}
word = ["h", "k", "u", "s", "t"]
for w in word:
    print(w, end="")
print() # Add new line at the end
# Output:|\pause| hkust
\end{minted}
\end{frame}

\begin{frame}[fragile]{Summary}
\vspace{-6pt}
\begin{block}{Lists}
Lists are represented with [ ] to hold multiple variables, where the $i^{\text{th}}$ item is at index $i - 1$.\\
If a list is called \texttt{l}, one can:
\begin{itemize}
    \item print the list with \texttt{print(l)}.
    \item get the length of \texttt{l} with \texttt{len(l)}.
    \item get/edit the element at index \texttt{i} with \texttt{l[i]}.
    \item append a value \texttt{v} to \texttt{l} with \texttt{l.append(v)}.
    \item insert a value \texttt{v} to \texttt{l} at index \texttt{i} with \texttt{l.insert(i, v)}.
    \item remove the first occurence of a value \texttt{v} with \texttt{l.remove(v)}.
    \item reverse the list with \texttt{l.reverse()}.
    \item count the occurence of value \texttt{v} with \texttt{l.count(v)}.
    \item get the index of the first occurence of a value \texttt{v} with \texttt{l.index(v)}.
    \item sort the list with \texttt{l.sort()}.
\end{itemize}
\end{block}
\end{frame}

\begin{frame}[fragile]{Summary}
\begin{block}{\texttt{while} loops}
\begin{minted}[tabsize=4, escapeinside=||]{python}
while condition:
    # Do code
\end{minted}
Code in the \texttt{while} block are run while the condition is fulfilled.\\
Do make sure that the \texttt{while} loop can be exited.
\end{block}
\end{frame}

\begin{frame}[fragile]{Summary}
\begin{block}{\texttt{for} loops and \texttt{range}}
\begin{minted}[tabsize=4, escapeinside=||]{python}
n = 5 # Example
for i in range(n):
    # Do code with each number from 0 to n - 1
\end{minted}
\texttt{range(n)} returns a range of integers that starts from 0 and ends at \texttt{n} - 1.
\end{block}

\begin{block}{\texttt{for} loops and lists}
\begin{minted}[tabsize=4, escapeinside=||]{python}
l = [...] # A list with items
for i in l:
    # Do code with each item in the list
\end{minted}
\texttt{for} loops can be directly applied onto lists.
\end{block}
\end{frame}

\begin{frame}{\text{}}
	\begin{center}
		The end\\
		Written in \LaTeX\\
		Last updated: 18 Mar 2024
	\end{center}
\end{frame}
\end{document}