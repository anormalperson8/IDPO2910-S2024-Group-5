\documentclass[dvipsnames, svgnames, x11names]{beamer}
\usepackage[T1]{fontenc}
\usepackage{bold-extra}
\usepackage{amsfonts, amsthm, amsmath, amssymb}
\usepackage{mathtools}
\usepackage{ulem}
\usepackage{bm}
\usepackage{minted}
\usemintedstyle{vs}
\usetheme{Madrid}
\usecolortheme{default}
\usepackage{multirow, multicol}
\usepackage{tikz, pgfplots}
\usetikzlibrary{arrows.meta, math, calc, quotes, intersections, angles, trees, shapes.geometric}
\pgfplotsset{compat=1.18}
\tikzstyle{block} = [rectangle, minimum width=4cm, minimum height=2cm, text centered, draw = black]
\tikzmath{\x1 = 2.5; \y1 = 3;}

\graphicspath{{../images}}

\title[HKUST Future-Ready Scholars]{HKUST Future-Ready Scholars}
\subtitle{Introduction to Game Programming using Python}
\author[Game Programming using Python]{Part 1: Number Guessing Game}
\date[April 2024]{20 April 2024}
\titlegraphic{\includegraphics[height=1cm]{ust.png}}

\begin{document}

\setbeamertemplate{page number in head/foot}{\insertframenumber /68}

\frame{\titlepage}

\begin{frame}[fragile]{Introduction}
    \begin{center}
        Open a tab on your browser, then go to

        \href{https://www.menti.com/}{https://www.menti.com/}

        \

        \frame{\includegraphics[width=0.75\textwidth]{Menti-Demo.png}}
    \end{center}
\end{frame}

\begin{frame}[fragile]{Number Guessing Game}
    \begin{center}
        Let's play the number guessing game.
    \end{center}
\end{frame}

\begin{frame}[fragile]{Google Colab}
    \begin{center}
        Set-up your Gmail account.

        \

        Then head to

        \href{https://colab.research.google.com/}{https://colab.research.google.com/}
    \end{center}
\end{frame}

\begin{frame}[fragile]{Files}
    \begin{center}
        All materials today are at:

        \href{https://bit.ly/ustidpo}{https://bit.ly/ustidpo}

        \

        \frame{\includegraphics[width=0.75\textwidth]{Download.png}}

        Download all files that belong to \textbf{Workshop 1} today.
    \end{center}
\end{frame}

\begin{frame}[fragile]{Jupyter Notebook}
    \begin{center}
        Now upload your Jupyter Notebook file with \textbf{Files $\rightarrow$ Open Notebook}.
    
        \

        \frame{\includegraphics[width=0.5\textwidth]{Upload.png}}

        Upload the file \textbf{Number-Guessing.ipynb}.
    
    \end{center}
\end{frame}

\begin{frame}[fragile]{Using Jupyter Notebook}
    You can type your code in these blocks. We call these blocks code cells.

    \begin{center}
        \frame{\includegraphics[width=0.5\textwidth]{Use1.png}}
    \end{center}

    \

    You can run a code cell with the button on the left.

    \begin{center}
        \frame{\includegraphics[width=0.5\textwidth]{Use2.png}}
    \end{center}
\end{frame}

\begin{frame}{World of Game Coding}
    \tikzstyle{block} = [rectangle, minimum width=4cm, minimum height=2cm, text centered, draw = black]
    \tikzmath{\x1 = 2.5; \y1 = 3;}
    
    \hspace{0.2\textwidth}\scalebox{0.5}{\begin{tikzpicture}
    \node (root) [block, fill = Cornsilk2] {Game (Software)};
    \node (program) [block, fill = LightSteelBlue] at ($(root) + (-\x1, -\y1)$) {Program (Codes)};
    \node (mm) [block, fill = Cornsilk2] at ($(root) + (\x1, -\y1)$) {Multimedia};
    
    \node (func) [block, fill = LightSteelBlue] at ($(program) + (0, -\y1)$) {Functions};
    
    \node (if) [block, fill = LightSteelBlue] at ($(func) + (-\x1, -\y1)$) {Decision Making};
    \node (loop) [block, fill = LightSteelBlue] at ($(func) + (\x1, -\y1)$) {Loops};
    
    \node (io) [block, fill = LightSteelBlue] at ($(if) + (0, -\y1)$) {Input/Output};
    \node (var) [block, fill = LightSteelBlue] at ($(loop) + (0, -\y1)$) {Variables};
    
    \draw [->] (root.south) -- ++(0, -0.5) -- ++(-\x1, 0) -- (program.north);
    \draw [->] (root.south) -- ++(0, -0.5) -- ++(\x1, 0) -- (mm.north);
    
    \draw [->] (program.south) -- (func.north);
    
    \draw [->] (func.south) -- ++(0, -0.5) -- ++(-\x1, 0) -- (if.north);
    \draw [->] (func.south) -- ++(0, -0.5) -- ++(\x1, 0) -- (loop.north);
    
    \draw [->] (loop.south) -- ++(0, -0.5) -- ++(-2*\x1, 0) -- (io.north);
    \draw [->] (if.south) -- ++(0, -0.5) -- ++(2*\x1, 0) -- (var.north);
    \end{tikzpicture}}
    
\end{frame}

\begin{frame}{What is Python?}
    Did you know? Python was made by someone who was bored.
    
    It's a language designed to be almost as understandable as English.
    
    You will be using Python 3. Why? Because Python 1 are 2 are too old.
    \begin{center}
    \includegraphics[height=1.5cm]{python-logo.png}\\
    This is the logo of Python.
    \end{center}
\end{frame}
    
\begin{frame}{Contents} 
\begin{center}
    \scalebox{0.5}{\begin{tikzpicture}
    \node (program) [block, fill = LightSteelBlue] {Program (Codes)};
    
    \node (func) [block, fill = LightSteelBlue] at ($(program) + (0, -\y1)$) {Functions};
    
    \node (if) [block, fill = LightSteelBlue] at ($(func) + (-\x1, -\y1)$) {Decision Making};
    \node (loop) [block, fill = LightSteelBlue] at ($(func) + (\x1, -\y1)$) {Loops};
    
    \node (io) [block, fill = DarkSeaGreen2] at ($(if) + (0, -\y1)$) {Input/Output};
    \node (var) [block, fill = LightSteelBlue] at ($(loop) + (0, -\y1)$) {Variables};
    
    \draw [->] (program.south) -- (func.north);
    
    \draw [->] (func.south) -- ++(0, -0.5) -- ++(-\x1, 0) -- (if.north);
    \draw [->] (func.south) -- ++(0, -0.5) -- ++(\x1, 0) -- (loop.north);
    
    \draw [->] (loop.south) -- ++(0, -0.5) -- ++(-2*\x1, 0) -- (io.north);
    \draw [->] (if.south) -- ++(0, -0.5) -- ++(2*\x1, 0) -- (var.north);
    \end{tikzpicture}}
\end{center}
\end{frame}

\begin{frame}[fragile]{The first thing in Python - \texttt{print()} function}

\begin{minted}[tabsize=4]{python} 
print("This is the print function.")
\end{minted}
\end{frame}
    
\begin{frame}[fragile]{The first thing in Python - \texttt{print()} function}
\texttt{print()} is a function that lets you print something,\\
also known as text output.
\begin{minted}[tabsize=4]{python}
print("Word") # This prints the word "Word".
\end{minted}
\vspace{1em}
Examples:
\begin{minted}[tabsize=4]{python} 
>>> print("Hello World")
Hello World
>>> print("Haha hehe")
Haha hehe
>>> print(5)
5
\end{minted}
\end{frame}

\begin{frame}[fragile]{Printing multiple things}
You can use a comma (\texttt{,}) to separate different things with a space.
\begin{minted}[tabsize=4]{python} 
>>> print("Alpha", "Beta", "Gamma")
Alpha Beta Gamma
>>> print("Haha", "hehe")
Haha hehe
>>> print(19, 91)
19 91
\end{minted}
\end{frame}

\begin{frame}{Contents}

\begin{center}\scalebox{0.5}{
\begin{tikzpicture}
\node (program) [block, fill = LightSteelBlue] {Program (Codes)};

\node (func) [block, fill = LightSteelBlue] at ($(program) + (0, -\y1)$) {Functions};

\node (if) [block, fill = LightSteelBlue] at ($(func) + (-\x1, -\y1)$) {Decision Making};
\node (loop) [block, fill = LightSteelBlue] at ($(func) + (\x1, -\y1)$) {Loops};

\node (io) [block, fill = LightSteelBlue] at ($(if) + (0, -\y1)$) {Input/Output};
\node (var) [block, fill = DarkSeaGreen2] at ($(loop) + (0, -\y1)$) {Variables};

\draw [->] (program.south) -- (func.north);

\draw [->] (func.south) -- ++(0, -0.5) -- ++(-\x1, 0) -- (if.north);
\draw [->] (func.south) -- ++(0, -0.5) -- ++(\x1, 0) -- (loop.north);

\draw [->] (loop.south) -- ++(0, -0.5) -- ++(-2*\x1, 0) -- (io.north);
\draw [->] (if.south) -- ++(0, -0.5) -- ++(2*\x1, 0) -- (var.north);
\end{tikzpicture}}
\end{center}

\end{frame}

\begin{frame}[fragile]{Variables}
\begin{center}
Imagine you borrow a box from the computer.

\includegraphics[height=3cm]{box}

\pause
Give it a name and a value, you can now recall this value with the name!
\end{center}
\end{frame}

\begin{frame}[fragile]{Variables}
The code usually goes:\\
\texttt{variable\char`\_name = data}\\
This means whatever \texttt{data} is, it is now stored in a variable with name \texttt{variable\char`\_name}.
\vspace{1em}\\
Some basic variable types:
\begin{minted}[tabsize=4, escapeinside=||]{python}
a = 5       |\pause|# This is an integer (int) stored in a |\pause|
b = True    |\pause|# This is a boolean (bool) stored in b |\pause|
c = 3.2     |\pause|# This is a float (float)  stored in c |\pause|
d = "abc"   |\pause|# This is a string (str)   stored in d |\pause|
e = 'abc'   |\pause|# This is also a string    stored in e
\end{minted}
\end{frame}

\begin{frame}[fragile]{Variables - Integers}
What are integers?\pause\\
Integers are just like what you've learnt in Maths, numbers without decimal points. Are the following valid?\pause
\begin{minted}[tabsize=4, escapeinside=||]{python}
a = 5       |\pause|# Valid|\pause|
b = 12      |\pause|# Valid|\pause|
c = 69420   |\pause|# Valid|\pause|
d = -1984   |\pause|# Valid|\pause|
e = 32.5    |\pause|# This would become a float instead|\pause|
f = '5'     |\pause|# This would become a string instead
\end{minted}
\end{frame}

\begin{frame}[fragile]{Variables - Integer Arithmetic Operations}
You can do normal operations on integers:
\begin{minted}[tabsize=4, escapeinside=||]{python}
a = 1 + 2   |\pause|# a stores the integer 3|\pause|
b = 80 - 52 |\pause|# b stores the integer 28|\pause|
c = 69 * -2 |\pause|# c stores the integer -138|\pause|
d = 6 / 4   |\pause|# d stores the float 1.5|\pause|
e = 18 / 2  |\pause|# e stores the float 9.0|\pause|
\end{minted}
\begin{alertblock}{Division in Python}
Whether a number can be precisely divided or not, division returns a \texttt{float}.
\end{alertblock}
\end{frame}

\begin{frame}[fragile]{Variables - Integer Arithmetic Operations}
Operations with variables:
\begin{minted}[tabsize=4, escapeinside=||]{python}
a = 100
b = 12 |\pause|
c = a + b   |\pause|# c stores the integer 112|\pause|
d = b - a   |\pause|# d stores the integer -88|\pause|
e = a * -b  |\pause|# e stores the integer -1200|\pause|
f = a / b   |\pause|# f stores the float 8.333333333333334
\end{minted}
\end{frame}

\begin{frame}[fragile]{Variables - Integer Arithmetic Operations}
Then how do we get an integer output?\pause
\begin{minted}[tabsize=4, escapeinside=||]{python}
a = 100
b = 12 |\pause|
c = a // b  |\pause|# c stores the integer 8|\pause|
            # // operator takes the closest and smaller 
            # integer from the division operation|\pause|
d = a % b   |\pause|# d stores the integer 4|\pause|
            # % operator takes the remainder of a 
            # division operation
\end{minted}
\end{frame}

\begin{frame}[fragile]{Variables - Integer Arithmetic Operations}
Also, the power (exponent) operation:
\begin{minted}[tabsize=4, escapeinside=||]{python}
a = 2
b = 5       |\pause|
c = a ** b  |\pause|# c stores the integer 32
            # ** operator means power
\end{minted}
\end{frame}

\begin{frame}[fragile]{Task 0}
    A short summary on operators and integers:
    \begin{block}{Examples of valid integers}
        \begin{minted}[autogobble, tabsize=4, escapeinside=||]{python}
        a = 5       
        b = 12      
        c = 1000000   
        d = -1984
        \end{minted}
    \end{block}
    \begin{block}{Arithmetic Operators}
        Some basic and commonly-used operators:\\
        \centering
        \begin{tabular}{rlrl}
        \texttt{+}:& add & \texttt{-}:& minus,\\
        \texttt{*}:& multiply & \texttt{/}:& divide,\\
        \texttt{//}:& quotient & \texttt{\%}:& remainder,\\
        \texttt{**}:& power &&
        \end{tabular}
    \end{block}
\end{frame}

\begin{frame}[fragile]{Task 0}
    \begin{center}
        Now go to your Number Guessing Game file's task 0.

        \

        Try replacing \texttt{???} with the right numbers/operators!
    \end{center}
\end{frame}
    
\begin{frame}[fragile]{Variables - Floats}
What are floats?\pause\\
Floats are numbers with decimal points.\pause\\
Arithmetic operators we learnt can be applied as well.
\begin{minted}[tabsize=4, escapeinside=||]{python}
a = 0.2     # a stores the float 0.2
b = 3.0     # b stores the float 3.0|\pause|
c = a + b   |\pause|# c stores the float 3.2|\pause|
d = b / a   |\pause|# d stores the float 15.0|\pause|
e = a ** b  |\pause|# e stores the float 0.008000000000000002|\pause|
\end{minted}
\begin{alertblock}{Inaccuracies}
Inaccuracies happen with decimals in Python. Be careful when dealing with floats.
\end{alertblock}
\end{frame}

\begin{frame}[fragile]{Variables - Floats}
What happens when you combine floats and integers? \pause
\begin{minted}[tabsize=4, escapeinside=||]{python}
a = 0.2     # a stores the float 0.2
b = 3       # b stores the integer 3|\pause|
c = a + b   # c stores the float 3.2
d = b / a   # d stores the float 15.0
e = a ** b  # e stores the float 0.008000000000000002
\end{minted}
\pause
\begin{block}{Arithmetic operations between \texttt{int} and \texttt{float}}
Arithmetic operations between integers and floats converts the integer into a float first before operating.
\end{block}
\end{frame}

\begin{frame}[fragile]{Variables - Boolean values}
What are boolean values?\pause\\
There are only 2 boolean values in existence: \texttt{True} and \texttt{False}.\pause
\begin{minted}[tabsize=4, escapeinside=||]{python}
a = True
b = False
\end{minted}


\vspace{1cm}
\begin{overlayarea}{\textwidth}{3cm}
\only<4>{That's it.}
\only<5>{Just kidding, we will elaborate more on boolean values later.}
\end{overlayarea}
\end{frame}

\begin{frame}[fragile]{Variables - Strings}
What are strings?\pause\\

\begin{minted}[tabsize=4, escapeinside=||]{python}
a = "word"  |\pause|# a stores the string "word"|\pause|
b = 'word2' |\pause|# b stores the string "word2"|\pause|
c = '5.20'  |\pause|# c stores the string "5.20"|\pause|
\end{minted}
\vspace{-0.275em}
\texttt{d = }{\color{BrickRed}\texttt{\textquotesingle abc"}}\pause \hspace{1.59em}{\color{Green}\texttt{\# error}} \pause
\begin{block}{Quotes}
In Python you must use corresponding quotation marks for strings.
\end{block}
\end{frame}

\begin{frame}[fragile]{Variables - Strings}
How do I put the symbols \texttt{\textquotesingle} and \texttt{"} into a string?\pause\\
For \texttt{"}:\pause
	
\begin{minted}[tabsize=4, escapeinside=||]{python}
a = "word\"" # a stores the string "word""|\pause|
b = 'word"'  # b stores the same string as a|\pause|
\end{minted}
\vspace{1em}
Same goes for single quotes \texttt{\textquotesingle}:
\begin{minted}[tabsize=4, escapeinside=||]{python}
a = 'word\'' # a stores the string "word'"
b = "word'"  # b stores the same string as a
\end{minted}
\end{frame}

\begin{frame}[fragile]{Variables - Strings}
There are additional symbols in strings.
\begin{minted}[tabsize=4, escapeinside=||]{python}
a = "word\n" # \n represents the newline character
b = "word\t" # \t represents the tab character
\end{minted}
\end{frame}

\begin{frame}[fragile]{Variables - Strings}
Example:
\begin{minted}[tabsize=4, escapeinside=||]{python}
a = "haha"
b = "hehe"
c = a + b    |\pause|# c stores the string "hahahehe"
\end{minted}
\pause
\begin{block}{Concatenation of strings}
You can concatenate (add) strings together with the addition symbol.
\end{block}
\end{frame}

\begin{frame}{Contents}
\begin{center}\scalebox{0.5}{
\begin{tikzpicture}
\node (program) [block, fill = LightSteelBlue] {Program (Codes)};

\node (func) [block, fill = LightSteelBlue] at ($(program) + (0, -\y1)$) {Functions};

\node (if) [block, fill = LightSteelBlue] at ($(func) + (-\x1, -\y1)$) {Decision Making};
\node (loop) [block, fill = LightSteelBlue] at ($(func) + (\x1, -\y1)$) {Loops};

\node (io) [block, fill = DarkSeaGreen2] at ($(if) + (0, -\y1)$) {Input/Output};
\node (var) [block, fill = DarkSeaGreen2] at ($(loop) + (0, -\y1)$) {Variables};

\draw [->] (program.south) -- (func.north);

\draw [->] (func.south) -- ++(0, -0.5) -- ++(-\x1, 0) -- (if.north);
\draw [->] (func.south) -- ++(0, -0.5) -- ++(\x1, 0) -- (loop.north);

\draw [->] (loop.south) -- ++(0, -0.5) -- ++(-2*\x1, 0) -- (io.north);
\draw [->] (if.south) -- ++(0, -0.5) -- ++(2*\x1, 0) -- (var.north);
\end{tikzpicture}}
\end{center}
\end{frame}

\begin{frame}[fragile]{Variables in output - Revisiting the \texttt{print()} function}
How do we print variables?

\begin{minted}[tabsize=4, escapeinside=||]{python}
a = 5
print(a)       |\pause|# 5 |\pause|
b = "haha"
print(b)       |\pause|# haha |\pause|
print(a + 2)   |\pause|# 7 |\pause|
print(b + "a") |\pause|# hahaa |\pause|
\end{minted}
\begin{block}{Calculation}
We can calculate expressions inside the \texttt{print()} function.
\end{block}
\end{frame}

\begin{frame}[fragile]{Variables in output - Revisiting the \texttt{print()} function}
How do we print variables?

\begin{minted}[tabsize=4, escapeinside=||]{python}
a = 5
print(a)       |\pause|# 5 |\pause|
b = "haha"
print(a, b)    |\pause|# 5 haha |\pause|
print(b, b)    |\pause|# haha haha |\pause|
\end{minted}
\begin{block}{The comma}
Using \texttt{,} in \texttt{print()} would add a space in between the 2 items.
\end{block}
\end{frame}

\begin{frame}[fragile]{Variables in output - Revisiting the \texttt{print()} function}
How do we print variables?

\begin{minted}[tabsize=4, escapeinside=||]{python}
a = 5
print(a)       # 5
b = "haha"
print(b)       # haha|\pause|
print(a + "5") |\pause|# error |\pause|
print(b + 2)   |\pause|# error |\pause|
print(a + b)   |\pause|# error |\pause|
\end{minted}
\begin{block}{Addition}
You cannot use addition to print things of incompatible types.\\
\texttt{int} and \texttt{float} types are not incompatible because all \texttt{int} are converted to \texttt{float} if needed during operation.
\end{block}
\end{frame}

\begin{frame}[fragile]{Variables in output - Revisiting the \texttt{print()} function}
How do we print variables?

\begin{minted}[tabsize=4, escapeinside=||]{python}
a = 5
b = 32
c = 32.0
print(a * b)    # 160
print(a * c)    # 160.0
\end{minted}
\pause
\begin{block}{Takeaway}
\texttt{print()} function evaluates the expression inside the brackets first before actually printing.
\end{block}
\end{frame}

\begin{frame}[fragile]{Task 1}
    \begin{center}
        Now go to your Number Guessing Game file's task 1.

        \
            
        All lines you have to do is marked with ``\texttt{\# TODO:}''.

        Try to finish the \texttt{print()} statements!
    \end{center}
\end{frame}

\begin{frame}[fragile]{\texttt{input()} function}
We know how to output (\texttt{print}), what about input?
\begin{minted}[tabsize=4]{python} 
input("This is the input function.")
\end{minted}
\end{frame}


\begin{frame}[fragile]{\texttt{input()} function}
\texttt{input()} is a function that outputs a prompt and lets the user enter something.
\begin{minted}[tabsize=4, escapeinside=||]{python} 
>>> input("Enter a number: ")
Enter a number: |{\color{ForestGreen}\textit{5}}|
\end{minted}
% VSCode highlighting workaround
\if 0
}
\end{minted}\fi

\vspace{1em}
Simply inputting doesn't do anything, but we can print it.
\begin{minted}[tabsize=4, escapeinside=||]{python} 
>>> print(input("Enter a number: "))
Enter a number: |{\color{ForestGreen}\textit{100}}|
100
\end{minted}
% VSCode highlighting workaround
\if 0
}
\end{minted}\fi
\end{frame}

\begin{frame}[fragile]{\texttt{input()} function}
Another example:
\begin{minted}[tabsize=4, escapeinside=||]{python} 
>>> input("Enter something: ")
Enter something: |{\color{ForestGreen}\textit{I am in HKUST}}|
\end{minted}
% VSCode highlighting workaround
\if 0
}
\end{minted}\fi

\vspace{1em}
Simply inputting doesn't do anything, but we can print it.
\begin{minted}[tabsize=4, escapeinside=||]{python} 
>>> print(input("Enter a number: "))
Enter something: |{\color{ForestGreen}\textit{I am in HKUST}}|
\end{minted}
\vspace{-0.275em}
\texttt{I am in HKUST}
% VSCode highlighting workaround
\if 0
}
\end{minted}\fi
\end{frame}

\begin{frame}[fragile]{Converting the type of an input}
How do we convert the data type of variables?
\begin{minted}[tabsize=4, escapeinside=||]{python} 
>>> number = input("Enter your number: ")
Enter your number: |{\color{ForestGreen}\textit{50}}|
>>> print(number) |\pause|
50 |\pause|
>>> print(number + 1000) |\pause|# Error occurs. Why?
\end{minted}
% VSCode highlighting workaround
\if 0
}
\end{minted}\fi
\pause
\begin{block}{Explanation}
\texttt{number} is a string type while 1000 is an integer.
\end{block}
\end{frame}

\begin{frame}[fragile]{Converting the type of an input}
How do we convert the data type of variables?
\begin{minted}[tabsize=4, escapeinside=||]{python} 
>>> number = input("Enter your number: ")
Enter your number: |{\color{ForestGreen}\textit{50}}|
>>> print(number) |\pause|
50 |\pause|
>>> print(int(number) + 1000) |\pause|# 1050
\end{minted}
% VSCode highlighting workaround
\if 0
}
\end{minted}\fi
\begin{block}{Type conversion}
\texttt{input()} returns the input as string. We need to convert the input to the suitable type when needed.\\
We use \texttt{int()} to convert something into an integer.\\
This will be useful in the number guessing game.
\end{block}
\end{frame}

\begin{frame}[fragile]{Task 2}
    \begin{center}
        Now go to your Number Guessing Game file's task 2.

        \
            
        All lines you have to do is marked with ``\texttt{\# TODO:}''.

        Try to finish the code to get the user's input.
    \end{center}
\end{frame}

\begin{frame}[fragile]{Generating a random integer using \texttt{random} library}
In Python, we can import libraries to help us with tasks. One of them is generating random numbers.
The library/package \texttt{random} allows us to get a random number.

The \texttt{randint} function provided allows us to generate a random integer given a range.
\begin{minted}[tabsize=4, escapeinside=||]{python}
import random
num = random.randint(1, 10) # generates a random number
                            # We passed 1 and 10 into randint,
                            # so the number can only be 
                            # from 1 to 10
print(num) # prints the number
\end{minted}
\end{frame}

\begin{frame}[fragile]{Generating a random integer using \texttt{random} library}
Another example:
\begin{minted}[tabsize=4, escapeinside=||]{python}
import random
min = 15
max = 30
print(random.randint(min, max)) # prints a random number 
                                # from 15 to 30
\end{minted}
\end{frame}

\begin{frame}{Contents}

\begin{center}\scalebox{0.5}{
\begin{tikzpicture}
\node (program) [block, fill = LightSteelBlue] {Program (Codes)};

\node (func) [block, fill = LightSteelBlue] at ($(program) + (0, -\y1)$) {Functions};

\node (if) [block, fill = DarkSeaGreen2] at ($(func) + (-\x1, -\y1)$) {Decision Making};
\node (loop) [block, fill = LightSteelBlue] at ($(func) + (\x1, -\y1)$) {Loops};

\node (io) [block, fill = LightSteelBlue] at ($(if) + (0, -\y1)$) {Input/Output};
\node (var) [block, fill = LightSteelBlue] at ($(loop) + (0, -\y1)$) {Variables};

\draw [->] (program.south) -- (func.north);

\draw [->] (func.south) -- ++(0, -0.5) -- ++(-\x1, 0) -- (if.north);
\draw [->] (func.south) -- ++(0, -0.5) -- ++(\x1, 0) -- (loop.north);

\draw [->] (loop.south) -- ++(0, -0.5) -- ++(-2*\x1, 0) -- (io.north);
\draw [->] (if.south) -- ++(0, -0.5) -- ++(2*\x1, 0) -- (var.north);
\end{tikzpicture}}
\end{center}

\end{frame}

\begin{frame}{Decision Making}
\begin{center}
What is decision making? \pause 

\

We use condition(s) to decide whether some code should be run.

\

\tikzstyle{startstop} = [rectangle, rounded corners, 
minimum width=3cm, 
minimum height=1cm,
text centered, 
draw=black, 
fill=red!30]

\tikzstyle{io} = [trapezium, 
trapezium stretches=true, % A later addition
trapezium left angle=70, 
trapezium right angle=110, 
minimum width=3cm, 
minimum height=1cm, text centered, 
draw=black, fill=blue!30]

\tikzstyle{process} = [rectangle, 
minimum width=3cm, 
minimum height=1cm, 
text centered, 
text width=3cm, 
draw=black, 
fill=orange!30]

\tikzstyle{decision} = [diamond, 
minimum width=3cm, 
minimum height=1cm, 
text centered, 
draw=black, 
fill=green!30]
\tikzstyle{arrow} = [thick,->,>=stealth]

\scalebox{0.4}{\begin{tikzpicture}[node distance=2cm]

\node (start) [startstop] {Start};
\node (in1) [io, right of=start, xshift=2cm] {Input};
\node (pro1) [process, below of=in1] {Process 1};
\node (dec1) [decision, below of=pro1, yshift=-0.5cm] {Decision 1};

\node (pro2a) [process, left of=dec1, xshift=-2cm] {Process 2a};

\node (pro2b) [process, right of=dec1, xshift=2cm] {Process 2b};
\node (out1) [io, below of=pro2a] {Output};
\node (stop) [startstop, right of=out1, xshift=2cm] {Stop};

\draw [arrow] (start) -- (in1);
\draw [arrow] (in1) -- (pro1);
\draw [arrow] (pro1) -- (dec1);
\draw [arrow] (dec1) -- node[anchor=south] {False} (pro2a);
\draw [arrow] (dec1) -- node[anchor=south] {True} (pro2b);
\draw [arrow] (pro2a) -- (out1);
\draw [arrow] (out1) -- (stop);
\draw [arrow] (pro2b) |- (stop);

\end{tikzpicture}}
\end{center}
\end{frame}

\begin{frame}[fragile]{The \texttt{if} clause}
\begin{minted}[tabsize=4, escapeinside=||]{python}
a = 5 # a stores the integer 5
if a == 5:
    print("a stores 5.") # This line is activated
    
b = 10 # b stores the integer 10
if b == 5:
    print("b stores 5.") # This line is not activated
\end{minted}
\begin{block}{The \texttt{if} clause}
If the condition is true, then the code under it is run.
\end{block}
\end{frame}

\begin{frame}[fragile]{The \texttt{==} operator}
\begin{minted}[tabsize=4, escapeinside=||]{python}
a = 5 # a stores the integer 5
if a == 5:
    print("a stores 5.") # This line is activated
    
b = 10 # b stores the integer 10
if b == 5:
    print("b stores 5.") # This line is not activated
\end{minted}
\begin{block}{The \texttt{==} operator}
The operator \texttt{==} is used to compare 2 values. If the values on the both sides are the same, then it becomes {\color{blue}\texttt{True}}. It becomes {\color{blue}\texttt{False}} otherwise.
\end{block}
\end{frame}

\begin{frame}[fragile]{The \texttt{if}-\texttt{else} clause}
\begin{minted}[tabsize=4, escapeinside=||]{python}
a = 5 # a stores the integer 5
if a == 5:
    print("a stores 5.") # This line is activated
else:
    print("a does not store 5.") 
    
b = 10 # b stores the integer 10
if b == 5:
    print("b stores 5.")
else:
    print("b does not store 5.") # This line is activated
\end{minted}
\begin{block}{The \texttt{else} statement}
Code under the \texttt{else} statement is executed when the condition in \texttt{if} is not true.
\end{block}
\end{frame}

\begin{frame}[fragile]{The \texttt{if}-\texttt{else} clause}
\begin{minted}[tabsize=4, escapeinside=||]{python}
a = 5 # a stores the integer 5
if a == 5:
|\textvisiblespace\textvisiblespace\textvisiblespace\textvisiblespace|print("a stores 5.") # This line is activated
else:
|\textvisiblespace\textvisiblespace\textvisiblespace\textvisiblespace|print("a does not store 5.") 
    
b = 10 # b stores the integer 10
if b == 5
|\textvisiblespace\textvisiblespace\textvisiblespace\textvisiblespace|print("b stores 5.")
else:
|\textvisiblespace\textvisiblespace\textvisiblespace\textvisiblespace|print("b does not store 5.") # This line is activated
\end{minted}
\begin{block}{Indentation in Python}
Indentation decides whether the code is under the \text{if}/\texttt{else} statements.\\
It does not have to be 4 spaces, but they have to be \textbf{consistent}.
\end{block}
\end{frame}

\begin{frame}[fragile]{The \texttt{if}-\texttt{elif}-\texttt{else} clause}
\begin{minted}[tabsize=4, escapeinside=||]{python}
a = 5 # a stores the integer 5
if a == 5:
    print("a stores 5.") # This line is activated
elif a == 10:
    print("a stores 10.")
else:
    print("a does not store 5 or 10.") 
\end{minted}
\begin{block}{The \texttt{elif} statement}
The \texttt{elif} (stands for else-if) statement is a secondary \text{if} statement that is run if the previous \texttt{if}/\texttt{elif} condition(s) are not true.
\end{block}
\end{frame}

\begin{frame}[fragile]{The \texttt{if}-\texttt{elif}-\texttt{else} clause}
\begin{minted}[tabsize=4, escapeinside=||]{python}
a = 15 # a stores the integer 15
if a == 5:
    print("a stores 5.") 
elif a == 10:
    print("a stores 10.")
elif a == 15:
    print("a stores 15") # This line is activated
else:
    print("a does not store 5, 10 or 15.") 
\end{minted}
\begin{block}{Stacking the \texttt{elif} statement}
The \texttt{elif} statement can be stacked on top of one another.
\end{block}
\end{frame}

\begin{frame}[fragile]{Comparison Operators}
We've learnt that \texttt{==} means ``equal to''. What are some other operators?
\begin{center}
\begin{tabular}{|c|c|}\hline
Operator & Meaning\\\hline
\texttt{==} & equal to\\\hline
\texttt{>} & larger than\\\hline
\texttt{>=} & larger than or equal to\\\hline
\texttt{<} & smaller than\\\hline
\texttt{<=} & smaller than or equal to\\\hline
\texttt{!=} & not equal to\\\hline
\end{tabular}
\end{center}
\end{frame}

\begin{frame}[fragile]{Decision Making and Comparison Operators}
\begin{minted}[tabsize=4, escapeinside=||]{python}
a = 10 # a stores the integer 10
if a > 5:
    print("a is larger than 5")

if a >= 10:
    print("a is larger than or equal to 10")
\end{minted}

\vspace{1em}
In this example, both \texttt{print()} statements are activated.
\end{frame}

\begin{frame}[fragile]{Decision Making and Comparison Operators}
\begin{minted}[tabsize=4, escapeinside=||]{python}
a = 10 # a stores the integer 10
if a > 5:
    print("a is larger than 5")
elif a >= 10:
    print("a is larger than or equal to 10") # Not run
\end{minted}

\pause
\vspace{1em}
In this example, only the first \texttt{print()} statements are activated.

\begin{block}{\texttt{if} vs \texttt{elif}}
If a condition is fulfilled, any \texttt{elif} clauses afterwards will not be considered.
\end{block}
\end{frame}

\begin{frame}[fragile]{Logic Operators - \texttt{and}}
The \texttt{and} operator denotes whether the 2 conditions are fulfilled \textbf{at the same time}.

\

Example:
\begin{minted}[tabsize=4, escapeinside=||]{python}
a = 10 # a stores the integer 10
if a > 5 and a < 9:
    print("a is between 5 and 9")
else:
    print("a is not between 5 and 9") # This line is run
\end{minted}
\end{frame}

\begin{frame}[fragile]{Logic Operators - \texttt{or}}
The \texttt{or} operator denotes whether \textbf{any} of the 2 conditions are fulfilled.

\

Example:
\begin{minted}[tabsize=4, escapeinside=||]{python}
a = 10 # a stores the integer 10
if a < 5 or a > 9:
    print("a is not between 5 and 9") # This line is run
else:
    print("a is between 5 and 9")
\end{minted}
\end{frame}

\begin{frame}[fragile]{Logic Operators - \texttt{not}}
The \texttt{not} operator reverses the condition.

\

Example:
\begin{minted}[tabsize=4, escapeinside=||]{python}
a = 10 # a stores the integer 10
if not a == 5: # Same as a != 5
    print("a is not 5") # This line is run
else:
    print("a is 5")
\end{minted}
\end{frame}

\begin{frame}[fragile]{Multiple Logic Operators}
\begin{minted}[tabsize=4, escapeinside=||]{python}
a = 10 # a stores the integer 10
if not a % 2 != 0 or a == 1: # Same as a % 2 == 0 or a == 1
    print("a is even or equal to 1")
else:
    print("a is odd and not equal to 1")
\end{minted}

\

\begin{minted}[tabsize=4, escapeinside=||]{python}
b = 10 # b stores the integer 10
if b == 5 and not b % 2 != 0: # Impossible condition
    print("b is 5 and somehow even?")
else:
    print("else statement")
\end{minted}
\end{frame}

\begin{frame}[fragile]{Multiple Logic Operator (out of control)}
We can use multiple logic operators together, but what about the rules?
\begin{minted}[tabsize=4, escapeinside=||]{python}
a = 10 # a stores the integer 10
if not a == 0 and a == 1 or a == 3 and a % 2 == 1:
    print("What is going on in the conditions?")
else:
    print("Else statement")
\end{minted}
\end{frame}

\begin{frame}[fragile]{Multiple Logic Operator (out of control)}
We add brackets () to make our conditions clear.
\begin{minted}[tabsize=4, escapeinside=||]{python}
a = 10 # a stores the integer 10
if (not a == 0 and a == 1) or (a == 3 and a % 2 == 1):
    print("Now the conditions are clearer")
else:
    print("Else statement")
\end{minted}
\pause
\begin{block}{Reminder}
If you ever use > 1 \texttt{and}/\texttt{or} operators, add brackets to keep track of what your conditions are.
\end{block}
\end{frame}

\begin{frame}[fragile]{Task 3}
    \begin{center}
        Now go to your Number Guessing Game file's task 3.

        \
            
        All lines you have to do is marked with ``\texttt{\# TODO:}''.

        \

        We have to decide whether a number is larger/smaller\\
        than the secret number in the number guessing game.

        \

        Try to finish the code that makes the decision.
    \end{center}
\end{frame}

\begin{frame}[fragile]{Running the game}
    Go to the code cell that runs the main game:

    \

    \includegraphics[width=\textwidth]{NumGuessMain.png}

    If you did the tasks above correctly, you should be able to run the number guessing game!
\end{frame}

\begin{frame}[fragile]{Summary}
\begin{block}{Variable types}
There are 4 basic variable types: \texttt{int}, \texttt{bool}, \texttt{float} and \texttt{str}.
\end{block}
\begin{block}{Arithmetic Operators}
Some basic and commonly-used operators:\\
\centering
\begin{tabular}{rlrl}
\texttt{+}:& add & \texttt{-}:& minus,\\
\texttt{*}:& multiply & \texttt{/}:& divide,\\
\texttt{//}:& quotient & \texttt{\%}:& remainder,\\
\texttt{**}:& power &&
\end{tabular}
\end{block}
\end{frame}

\begin{frame}[fragile]{Summary}
\begin{block}{The \texttt{print()} statement}
\mint{py}{print(*objects)}

\fbox{\texttt{*objects}} - the things you want to print
\end{block}

\begin{block}{The \texttt{input()} statement}
\mint{py}{input(prompt)}
where \fbox{\texttt{prompt}} is quite literally what it means. It prints the output, then returns the value inputted as a string.
\end{block}
\begin{block}{\texttt{random.randint()}}
\mint{py}{random.randint(a, b)}
\fbox{\texttt{a}} - the lower bound of your range\\[1pt]
\fbox{\texttt{b}} - the upper bound of your range\\
This generates an integer \texttt{n} where \texttt{a} $\leq$ \texttt{n} $\leq$ \texttt{b}.
\end{block}
\end{frame}

\begin{frame}[fragile]{Summary}
\begin{block}{\texttt{if}, \texttt{elif} and \texttt{else}}
\texttt{if}, \texttt{elif} and \texttt{else} clauses are used to decide whether some code should be executed.
Whenever one is fulfilled, all others are ignored.
\begin{minted}[tabsize=4, escapeinside=||]{python}
if condition1: # if condition1 is true
    # Do something, ignore all elif and else below

elif condition2: # if condition2 is true
    # Do something, ignore all elif and else below

elif condition3: # if condition3 is true
    # Do something, ignore all elif and else below

else: # if all the conditions above are false
    # Do something
\end{minted}
\end{block}
\end{frame}

\begin{frame}[fragile]{Summary}
\begin{block}{The \texttt{and} logic operator}
The \texttt{and} operator makes it so that both conditions have to be fulfilled in order for the code it is under to execute.
\end{block}
\begin{block}{The \texttt{or} logic operator}
The \texttt{or} operator makes it so that only 1 of the conditions have to be fulfilled in order for the code it is under to execute.
\end{block}
\begin{block}{The \texttt{not} logic operator}
The \texttt{not} operator reverses the condition is it attached to.
\end{block}
\begin{alertblock}{Multiple logic operators}
One can chain multiple logic operators together, but to be safe add brackets () to make sure the condition works as intended.
\end{alertblock}
\end{frame}

\begin{frame}{ \ }
	\begin{center}
		The End\\
		Thank you!
	\end{center}
\end{frame}






%%%-------------------------ADDITIONAL CONTENT-------------------------%%%
\setcounter{framenumber}{0}
\setbeamertemplate{page number in head/foot}{S-\insertframenumber /S-16}











\begin{frame}{Additional content}
	\begin{center}
		Here are some additional content that we didn't have time to mention in the workshop.
	\end{center}
\end{frame}

\begin{frame}[fragile]{More on \texttt{print()} function}
In Python, the \texttt{print()} function automatically adds a new line after execution. We, however, can stop that.\\
The \texttt{end=} tag allows us to define the character added when \texttt{print()} is executed.
\begin{minted}[tabsize=4, escapeinside=||]{python}
print(5, end="")
print(4)
print("a", end="abc")
print("d", end=" ")
print("e")|\pause|
# What is the output?|\pause|
# Output: 54
#         aabcd e
\end{minted}
\begin{block}{End of line}
Remember to include a new line {\color{BrickRed}\texttt{\textbackslash n}} in the last line of a printed string.\\
Else it may mess up the future outputs from other lines of the code or the computer terminal.
\end{block}
\end{frame}

\begin{frame}[fragile]{More on \texttt{print()} function}
We mentioned that whenever \texttt{,} is used in \texttt{print()}, the items would be separated by a space.\\
This can actually be changed using the \texttt{sep=} tag.\pause
\begin{minted}[tabsize=4, escapeinside=||]{python}
>>> print("100", 100, end="\n3\n")|\pause|
>>> 100 100
	3|\pause|
>>> print("100", 100, sep="a", end="\n3\n")|\pause|
>>> 100a100
	3
\end{minted}

\end{frame}

\begin{frame}[fragile]{More on \texttt{print()} function}
Another example:
\begin{minted}[tabsize=4, escapeinside=||]{python}
>>> a = 5
>>> b = 10
>>> print(a, b, a + b, end="20\n")|\pause|
>>> 5 10 1520 |\pause|
>>> print(a, b, a + b, sep="", end="20\n")|\pause|
>>> 5101520	
>>> print(a, b, a + b, end="20\n", sep="")|\pause|
>>> 5101520	|\pause|
\end{minted}
\begin{block}{Command Parameters}
As long as you mark \texttt{sep} and \texttt{end} clearly \textbf{and} after the things you want to print, the ordering doesn't matter!
\end{block}
\end{frame}

\begin{frame}{Converting between types}
You can convert between types with their type names in Python.
\begin{center}
	\begin{tabular}{|c|c|}\hline
		Data Type & Command\\\hline
		Integer & \texttt{int()}\\\hline
		Float & \texttt{float()}\\\hline
		String & \texttt{str()}\\\hline
		Boolean & \texttt{bool()}\\\hline
	\end{tabular}
\end{center}
\end{frame}

\begin{frame}[fragile]{\texttt{int()}}
\texttt{int()} tries to convert a variable into an integer.
\begin{minted}[tabsize=4, escapeinside=||]{python} 
a = 10        # int
print(int(a)) |\pause|# 10
			  |\pause|# Nothing occurs |\pause|
			  
b = 3.7       # float
print(int(b)) |\pause|# 3
              |\pause|# Discards values to the right of 
              # the decimal point |\pause|
              
c = True      # boolean
print(int(c)) |\pause|# 1 |\pause|

d = False     # boolean
print(int(d)) |\pause|# 0 |\pause|
              # For boolean: 0 if False, True otherwise
              
\end{minted}
\end{frame}

\begin{frame}[fragile]{\texttt{int()}}
\begin{minted}[tabsize=4, escapeinside=||]{python}
i = "123abc"  # string
print(int(i)) |\pause|# Error |\pause|

j = "123"     # string with ONLY numbers
print(int(j)) |\pause|# 123 |\pause|
              # Only integers in strings would be 
              # successfully converted |\pause|

k = "123.123" # string with ONLY numbers, but with 
              # a number that represents a float
print(int(k)) |\pause|# Error 
\end{minted}
\end{frame}

\begin{frame}[fragile]{\texttt{float()}}
The concepts of \texttt{int()} and \texttt{float()} are quite similar. \pause
\begin{minted}[tabsize=4, escapeinside=||]{python} 
a = 10          # int
print(float(a)) |\pause|# 10.0
                |\pause|# From int -> float |\pause|

b = 3.7         # float
print(float(b)) |\pause|# 3.7
                |\pause|# Nothing happens |\pause|

c = True        # boolean
print(float(c)) |\pause|# 1.0 |\pause|

d = False       # boolean
print(float(d)) |\pause|# 0.0
\end{minted}
\end{frame}

\begin{frame}[fragile]{\texttt{float()}}
\begin{minted}[tabsize=4, escapeinside=||]{python}
i = "123abc"    # string
print(float(i)) |\pause|# Error |\pause|
 
j = "123"       # string with ONLY numbers
print(float(j)) |\pause|# 123.0 |\pause|

k = "123.123"   # string with ONLY numbers, but with 
                # a number that represents a float
print(float(k)) |\pause|# 123.123
\end{minted}
\end{frame}

\begin{frame}[fragile]{\texttt{str()}}
All of the 3 data types below can be transformed into strings. \pause
\begin{minted}[tabsize=4, escapeinside=||]{python} 
a = 10        # int
print(str(a)) |\pause|# 10 |\pause|

b = 3.7       # float
print(str(b)) |\pause|# 3.7 |\pause|

c = True      # boolean
print(str(c)) |\pause|# True |\pause|

d = False     # boolean
print(str(d)) |\pause|# False
\end{minted}
\end{frame}

\begin{frame}[fragile]{\texttt{str()}}
\begin{minted}[tabsize=4, escapeinside=||]{python} 
e = "abcdef"   # string
print(str(e))  |\pause|# abcdef |\pause|
               # Nothing happens
\end{minted}
\end{frame}

\begin{frame}[fragile]{\texttt{bool()}}
\begin{minted}[tabsize=4, escapeinside=||]{python} 
a = 0          # int
print(bool(a)) |\pause|# False
               |\pause|# 0 means False |\pause|

b = 3.7        # float
print(bool(b)) |\pause|# True |\pause|
\end{minted}

\begin{block}{True and False values}
	Any integers or floats, if they are not zero, then \texttt{bool()} returns \texttt{True}, \texttt{False} otherwise.
\end{block}

\pause

\begin{minted}[tabsize=4, escapeinside=||]{python} 
c = True       # boolean
print(bool(c)) # True

d = False      # boolean
print(bool(d)) # False
               |\pause|# Nothing happens for the 2 above
\end{minted}
\end{frame}

\begin{frame}[fragile]{\texttt{bool()}}
\texttt{bool()}, when applied to a string, checks whether it has content: \pause
\begin{minted}[tabsize=4, escapeinside=||]{python} 
e = "abcdefg"  
print(bool(e)) # True
f = "False"    
print(bool(f)) # True
g = " tRuE "   
print(bool(g)) # True
h = "0"        
print(bool(h)) # True
i = ""         
print(bool(i)) # False
\end{minted}

\begin{block}{Strings}
If the string has a length $>$ 0, then \texttt{bool()} returns \texttt{True}, \texttt{False} otherwise.
\end{block}

\end{frame}

\begin{frame}[fragile]{Example of input and type conversion}
\begin{minted}[tabsize=4, escapeinside=||]{python} 
age = int(input("How old are you? "))
print("You are", age, "years old.")
\end{minted}
\hspace{0pt}\pause

Running the program:\\

\begin{minted}[tabsize=4, escapeinside=||]{python} 
How old are you|$?$| |{\color{ForestGreen}\textit{69}}\pause|
You are 69 years old.
\end{minted}
% VSCode highlighting workaround
\if 0
}
\end{minted}\fi
\hspace{0pt}\pause

\begin{block}{Invalid input}
	If the input does not contain \textit{only} an integer, then the program would throw an error.
\end{block}

\end{frame}

\begin{frame}[fragile]{Example of input and type conversion}
\begin{minted}[tabsize=4, escapeinside=||]{python} 
age = int(input("How old are you? "))
print("You are", age, "years old.")
\end{minted}
\hspace{0pt}

Running the program with an invalid input:\\

\begin{minted}[tabsize=4, escapeinside=||]{python} 
How old are you|$?$| |{\color{ForestGreen}\textit{69.420}}|
\end{minted}
% VSCode highlighting workaround
\if 0
}
\end{minted}\fi
\pause
\begin{minted}[tabsize=4, escapeinside=||]{text}
Traceback (most recent call last):
    File "<stdin>", line 1, in <module>
ValueError: invalid literal for int() with base 10: '69.420'
\end{minted}
\hspace{0pt}\pause

\begin{block}{Invalid input}
This also applies to data types like boolean values and strings.
\end{block}
\end{frame}

\begin{frame}[fragile]{Summary}
\begin{block}{The \texttt{print()} statement}
\mint{py}{print(*objects, sep=' ', end='\n', file=None, flush=False)}

\

\fbox{\texttt{*objects}} - the things you want to print,\\
\fbox{\texttt{sep}} - the string that separates \texttt{objects} (when using commas),\\
\fbox{\texttt{end}} - the string to end the print statement with.\\
The other arguments can be ignored as they are rarely used.
\end{block}
\begin{block}{Type Conversion}
To convert between types, you can simply surround the target with brackets, and call the type.\\
\texttt{int -> int()}; \texttt{bool -> bool()}; \texttt{float -> float()}; \texttt{str -> str()}.
\end{block}
\end{frame}

\begin{frame}{ \ }
	\begin{center}
		End of Additional Contents\\
		Made in \LaTeX\\
		Last updated: 2 Apr 2024
	\end{center}
\end{frame}

\end{document}
