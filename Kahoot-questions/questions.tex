\documentclass[dvipsnames, svgnames, x11names, handout]{beamer}
\usepackage[T1]{fontenc}
\usepackage{bold-extra}
\usepackage{amsfonts, amsthm, amsmath, amssymb}
\usepackage{mathtools}
\usepackage{ulem}
\usepackage{minted}
\usemintedstyle{vs}
\usetheme{Madrid}
\usecolortheme{default}

\graphicspath{{../images}}

\title[HKUST Future-Ready Scholars]{HKUST Future-Ready Scholars}
\subtitle{Introduction to Game Programming using Python}
\author[Game Programming using Python]{}
\date[May 2024]{4 May 2024}
\titlegraphic{\includegraphics[height=1cm]{ust.png}}

\begin{document}


\frame{\titlepage}

\setbeamertemplate{footline}{}

\begin{frame}[fragile]{}
    \begin{center}
        \begin{minted}[autogobble, tabsize=4]{py}
            a = 1234
            b = 8888
        \end{minted}
    \end{center}
\end{frame}

\begin{frame}[fragile]{}
    \begin{center}
        \begin{minted}[autogobble, tabsize=4]{py}
            a = 120
            b = 12
            c = 1.2
            d = 112
        \end{minted}
    \end{center}
\end{frame}

\begin{frame}[fragile]{}
    \begin{center}
        \begin{minted}[autogobble, tabsize=4]{py}
            a = "abc"
            b = 'abc'
            c = "abcd"
            d = "abcd'
        \end{minted}
    \end{center}
\end{frame}

\begin{frame}[fragile]{}
    \begin{center}
        \begin{minted}[autogobble, tabsize=4]{py}
            a = '5'
            b = 5
            c = "5"
        \end{minted}
    \end{center}
\end{frame}

\begin{frame}[fragile]{}
    \begin{center}
        \begin{minted}[autogobble, tabsize=4]{py}
            a = 5 + 3
            b = 4 * 2
            c = -2 + 6
            d = 7 - (-1)
        \end{minted}
    \end{center}
\end{frame}

\begin{frame}[fragile]{}
    \begin{center}
        \begin{minted}[autogobble, tabsize=4]{py}
            print("I love", "HKUST.")
        \end{minted}
    \end{center}
\end{frame}

\begin{frame}[fragile]{}
    \begin{center}
        \begin{minted}[autogobble, tabsize=4, linenos, xleftmargin=1em]{py}
            a = 50
            b = 3
            print(a + b)
            print("53")
            print("5", "3")
        \end{minted}
    \end{center}
\end{frame}

\begin{frame}[fragile]{}
    \begin{center}
        \begin{minted}[autogobble, tabsize=4]{py}
            x = input("Enter a phrase: ")
            print(x)
        \end{minted}
    \end{center}
    Input: \texttt{Hello!}
\end{frame}

\begin{frame}[fragile]{}
    \begin{center}
        \begin{minted}[autogobble, tabsize=4]{py}
            x = input("Enter a phrase: ")
            print(x, "F")
        \end{minted}
    \end{center}
    Input: \texttt{ABCDE}
\end{frame}

\begin{frame}[fragile]{}
    \begin{center}
        \begin{minted}[autogobble, tabsize=4]{py}
            import random
            a = 5
            b = 15
            num = random.randint(a, b + 3)
            print(num)
        \end{minted}
    \end{center}
\end{frame}

\begin{frame}[fragile]{}
    \begin{center}
        \begin{minted}[autogobble, tabsize=4]{py}
            import random
            num = random.randint(2, 5)
            print(num)
        \end{minted}
    \end{center}
\end{frame}

\begin{frame}[fragile]{}
    \begin{center}
        \begin{minted}[autogobble, tabsize=4]{py}
            a = SOME_INTEGER # Any integer
            if ANSWER:
                print("TRUE")
            else:
                print("FALSE")
            
            not (a <= 0 or a > 100)    # (1)
            not a <= 0 and not a > 100 # (2)
            not (a <= 0 and a > 100)   # (3)
            not a <= 0 and a > 100     # (4)
        \end{minted}
    \end{center}
\end{frame}

\end{document}